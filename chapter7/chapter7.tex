\documentclass[../main.tex]{subfiles}

\begin{document}
\chapter{Fourierreihen}
Die Theorie der Fourierreihen wurde von Joseph Fourier
(1768--1830) entwickelt.
Die hier behandelte Theorie kann auch im Kapitel~XVII
in~\cite{heuser} nachgeschlagen werden.


\begin{motivation}
  Eine Gitarrensaite schwingt mit einer Grundschwingung
  \[
    u_0(x, t) = c_0 \cdot \sin(x) \cdot \cos(t)
  \]
  und Obertönen
  \[
    u_n(x, t) = c_{n+1} \cdot \sin(nx) \cdot \cos(nt),
  \]
  siehe Abbildung~\ref{fig:guitar}.
  Die $u_n$ sind Lösungen der Wellengleichung
  \[
    \frac{\partial^2}{\partial x} u(x, t)
    =
    \frac{\partial^2}{\partial t^2} u(x, t).
  \]
  Das Superpositionsprinzip sagt, dass auch
  $\sum_{k=1}^{n} u_k(x, t)$ eine Lösung
  mit Anfangsprofil 
  \[
    u_0(x, 0) + u_1(x, 0) + \cdots u_n(x, 0)
    = c_0 \sin(x) + c_1 \sin(2x) + \cdots + c_n \sin(nx)
  \]
  ist.
  Wir fragen uns nun, ob wir im Limes $n \to \infty$ 
  jedes Anfangsprofil so ausdrücken können.
  Die physikalische Intuition sagt uns, dass das gehen muss,
  doch es gibt stetige Funktionen auf $[0, \pi]$
  deren Graph unendliche Länge
  hat.
  Hier haben wir keine physikalische Interpretation des
  Anfangsprofils als Gitarrensaite. Die Theorie, die wir hier
  entwickeln, wird jedoch auch für solche Funktionen
  eine positive Antwort liefern.

\begin{figure}[htb]
  \centering
  \includegraphics{images/guitar}
  \caption{Eine Gitarrensaite mit weisser Grundschwingung
  und einigen grau eingezeichneten Obertönen}%
  \label{fig:guitar}
\end{figure}
\end{motivation}

\section{Definition der Fourierreihe}

Sei $f \colon \mathbb{R} \to \mathbb{R}$ stetig
und $2 \pi $-periodisch, das heisst, für alle
$x \in \mathbb{R}$ gilt $f(x + 2 \pi ) = f(x)$.
Dann ist $f$ durch die Einschränkung
$f|_{[0, 2\pi]} \colon [0, 2\pi] \to \mathbb{R}$ bestimmt.
Umgekehrt liefert jede Stetige Funktion
$f \colon[0, 2\pi] \to \mathbb{R}$ mit $f(0) = f(2\pi)$
eine stetige, $2\pi$-periodische Funktion
$f \colon \mathbb{R} \to \mathbb{R}$.
Dementsprechend werden wir in diesem Kapitel
den Blickwinkel laufend wechseln. Das sollte aber nicht
allzu viel Verwirrung stiften.

Wir betrachten den Ansatz
\[
  FR f(x) = a_0 + \sum_{k=1}^{\infty} a_k \cos(kx)
  + \sum_{k=1}^{\infty} b_k \sin(kx)
\]
für die Fourierreihe von einer
$2\pi$-periodischen Funktion $f \colon \mathbb{R} \to \mathbb{R}$.

\begin{remark}
  Bei der Gitarrensaite hatten wir nur
  $f|_{[0, \pi]}$ betrachtet. Ausserdem hatten wir die
  Nebenbedingungen $f(0) = f(\pi) = 0$.
  Deshalb hatten wir in diesem Szenario keine Cosinusterme.
\end{remark}

Wir versuchen nun, die Koeffizienten der Fourierreihe zu bestimmen.
Wir möchten, dass $f(x) = FR f(x)$ gilt. Dazu sollte
\[
  \int_{0}^{2\pi} f(x) \, dx 
  = \int_{0}^{2\pi} a_0 +  \sum_{k=1}^{\infty} a_k \cos(kx)
  + \sum_{k=1}^{\infty} b_k \sin(kx)
  \, dx
\]
gelten. Naiv erhalten wir nach Vertauschen des Limes mit dem Integral, dass
\[
  \int_{0}^{2\pi} f(x) \, dx =
  \int_{0}^{2\pi} a \, dx + \sum_{k=1}^{\infty} \int_{0}^{2\pi} a_k \cos (kx)
  + b_k \sin(kx)\, dx = 2\pi a_0.
\]
Wir stellen also die Vermutung auf, dass
\[
  a_0 = \frac{1}{2\pi} \int_{0}^{2\pi} f(x) \, dx
\]
gelten sollte.

\begin{lemma*}
  Seien $n, m \in \mathbb{N}$ mit $n, m \geq 1$.
  Dann gilt
  \begin{enumerate}[\normalfont(i)]
    \item $\int_{0}^{2\pi} \cos(mx) \sin(nx) \, dx = 0$,
    \item $\int_{0}^{2\pi} \cos(mx) \cos(nx) \, dx = 0$ falls $n \neq m$,
    \item $\int_{0}^{2\pi} \sin(mx) \sin(nx) \, dx = 0$ falls $n \neq m$,
    \item $\int_{0}^{2\pi} {\cos(nx)}^2 \, dx = \pi$,
    \item $\int_{0}^{2\pi} {\sin(nx)}^2 \, dx = \pi$.
  \end{enumerate}
\end{lemma*}


In der Sprache der linearen Algebra 
heisst das, dass die trigonometrischen Funktionen
$x \mapsto \cos(mx)$ und $x \mapsto \sin(nx)$
bezüglich des inneren Produkts
\[
  (f, g) = \int_{0}^{2\pi} f(x) g(x) \, dx
\]
orthogonal sind.
Wir wollen zeigen, dass das System dieser Funktionen
eine Orthogonalbasis ist.

\begin{proof}[Beweis vom Lemma]
  Benutze die Additionstheoreme, zum Beispiel
  \[
    \cos(x + y) = \cos x \cdot \cos y - \sin x \cdot \sin y.
  \]
  Für (i) berechnen wir
  \[
    \int_{0}^{2\pi} \cos(mx) \sin(nx) \, dx
     = \int_{-\pi}^{\pi} \cos(mx) \cdot \sin(nx) \, dx = 0,
\]
  da
  \[
    \int_{\pi}^{2\pi} \cos(mx) \sin(nx) \, dx
    = \int_{-\pi}^{0} \cos(mx) \sin(nx) \, dx
  \]
  gilt, und der Integrand ``antisymmetrisch'' ist, das heisst eine
  Funktion $f \colon \mathbb{R} \to \mathbb{R}$ ist mit
  $f(-x) = -f(x)$.
  Für (ii), verwende das Additionstheorem 
  \[
    \cos(mx) \cdot \cos(nx) = 1/2 \cdot \cos((n+m)x) + 1/2 \cdot \cos ((n-m)x)
  \]
  um zu zeigen, dass
  \[
    \int_{0}^{2\pi} \cos(mx) \cos(nx) \, dx = 0
  \]
  sobald $n \neq m$. Punkt (iii) ist analog.
  Für den Fall (iv) bemerke, dass
  \[
    {\cos(nx)}^2 = 1/2 \cdot \cos(2nx) + 1/2
  \]
  aus obigem Additionstheorem folgt. Somit gilt
  \[
    \int_{0}^{2\pi} {\cos(nx)}^2 \, dx = 1/2 \cdot 2 \pi = \pi.
  \]
  Punkt (v) ist ähnlich, nachdem wir ${\sin(nx)}^2 = 1 - {\cos(nx)}^2$ anwenden.
\end{proof}

Sei nun $n \geq 1$.
Berechne wieder naiv, dass
\begin{align*}
  \int_{0}^{2\pi} f(x) \cdot \cos(nx) \, dx
  &= \int_{0}^{2\pi} a_0 + \sum_{k=1}^{\infty} a_k \sin(kx) + b_k \cos(kx) \, d  \\
  &= a_0 \int_{0}^{2\pi} \cos(nx) \, dx
  + \sum_{k=1}^{\infty} \int_{0}^{2\pi} a_k \cos(kx)\cos(nx) \, dx\\
  &\;\;\;\;\;\;  + \sum_{k=1}^{\infty} \int_{0}^{2\pi} b_k \sin(kx) \cos(nx) \, dx\\
  &= a_n \pi,
\end{align*}
also sollte
\[
  a_n = \frac{1}{\pi} \int_{0}^{2\pi} f(x) \cos(nx) \, dx.
\]
gelten.
Analog sollte
\[
  b_n = \frac{1}{\pi}\int_{0}^{2\pi} f(x) \sin(nx) \, dx
\]
gelten. Diese naive Berechnung der Koeffizienten werden wir von nun
an als die Definition der Fourierreihe $FRf$ von $f$ betrachten.

\begin{example}
  Betrachte die Funktion
  \begin{align*}
    f \colon \mathbb{R} & \to \mathbb{R} \\
    x & \mapsto 
    \begin{cases}
      x & x \in (- \pi, \pi)\\
      0 & x = \pi,
    \end{cases}
  \end{align*}
  $2\pi$-periodisch fortgesetzt, siehe Abbildung~\ref{fig:linear}.

\begin{figure}[htb]
  \centering
  \includegraphics{images/linear}
  \caption{Die Funktion $f(x) = x$ fortgesetzt auf eine
  $2\pi$-periodische Funktion}%
  \label{fig:linear}
\end{figure}

  Berechne
  \[
    a_0 = \frac{1}{2\pi}\int_{0}^{2\pi} f(x) \, dx
    = \frac{1}{2\pi} \int_{-\pi}^{\pi} f(x) \, dx = 0.
  \]
  Berechne weiterhin
  \[
    a_n = \frac{1}{\pi} \int_{0}^{2\pi} f(x) \cos(nx) \, dx
    = \frac{1}{\pi} \int_{-\pi}^{\pi} f(x) \cos(nx) \, dx = 0.
  \]
  Weiter ist
  \begin{align*}
    b_n
    & = \frac{1}{\pi} \int_{0}^{2\pi} f(x) \sin(nx) \, dx\\
    &= \frac{1}{\pi} \int_{-\pi}^{\pi} x \sin(nx) \, dx \\
    &= \frac{1}{\pi}
    {\left[ \frac{-x \cos(nx)}{n} + \frac{\sin(nx)}{n^2} \right]}_{-\pi}^{\pi}\\
    &= -\frac{2}{n} \cdot {(-1)}^n.
  \end{align*}
  Daraus erhalten wir die Fourierreihe
  \[
    FRf(x) = \sum_{n=1}^{\infty} \frac{2}{n} \cdot {(-1)}^{n + 1} \cdot \sin(nx).
  \]
  Im Spezialfall $x = \pi/2$ erhalten wir
  \[
    FRf(\pi/2) = 2 \cdot (1 - 1/3 + 1/5 - 1/7 + \cdots).
  \]
  Falls wir zeigen könnten, dass die Funktion $f$ mit ihrer Fourierreihe
  übereinstimmt, würde folgen, dass
  \[
    \frac{\pi}{4} = 1 - \frac{1}{3} + \frac{1}{5} - \frac{1}{7} + \cdots.
  \]
\end{example}

Die Frage die sich aufdrängt ist, für welche $x \in \mathbb{R}$ 
die Gleichung $f(x) = FRf(x)$ gilt. Dies setzt die Konvergenz von
$FRf(x)$ voraus.

\begin{remark}
  Da $FRf$ immer $2\pi$-periodisch ist, reicht es die Frage
  für $x \in [0, 2\pi]$ zu beantworten.
\end{remark}

\begin{theorem}[Dirichlet 1829]
  Sei $f \colon \mathbb{R} \to \mathbb{R}$ stetig differenzierbar
  und $2\pi$-periodisch.
  Dann konvergiert $FRf(x)$ in allen Punkten $x \in \mathbb{R}$ 
  mit Grenzwert $f(x)$.
\end{theorem}

Dirichlet hat sogar gezeigt, dass beschränkte Variation reicht,
das heisst, dass der Funktionsgraph von $f$ über dem
Intervall $[0, 2\pi]$ endliche Länge hat.
Stetig differenzierbare Funktionen erfüllen diese Voraussetzung.
Die Originalquelle~\cite{dirichlet} ist überraschend gut
lesbar.
Von diesem Satz nicht abgedeckt werden Funktionen
wie in Abbildung~\ref{fig:long-graph}.

\begin{figure}[htb]
  \centering
  \includegraphics{images/long-graph}
  \caption{Eine Funktion mit unendlich langem Graph}%
  \label{fig:long-graph}
\end{figure}

\begin{theorem}[Fej\'er 1900]\label{thm:fejer}
  Sei $f \colon\mathbb{R} \to \mathbb{R}$ 
  stetig und $2 \pi$-periodisch.
  Dann konvergiert die Cesàro-gemittelte
  Fourierreihe
  gleichmässig auf $\mathbb{R}$ gegen $f$.
\end{theorem}

Für die Definition der Cesàro-gemittelten
Fourierreihe, siehe unten.
Auch hier ist der Originalartikel~\cite{fejer} gut lesbar.
Dieses Theorem werden wir später beweisen.

\begin{theorem}[Carleson 1966]
   Sei $f \colon \mathbb{R} \to \mathbb{R}$ 
   stetig und $2\pi$-periodisch.
   Dann existiert eine Lebesgue Nullmenge
   $\Delta \subset \mathbb{R}$, so dass die Fourierreihe
   $FRf(x)$ für alle $x \in \mathbb{R} \setminus \Delta$ 
   gegen $f(x)$ konvergiert.
\end{theorem}

Für den Beweis dieses Satzes in~\cite{carleson}
hat Carleson im Jahre 2006 den Abelpreis erhalten.
Das Resultat ist sehr schwierig.

\section{Der Fejérkern}
Sei $f \colon \mathbb{R} \to \mathbb{R}$ stetig
und $2\pi$-periodisch.
Berechne
\begin{align*}
  FRf(x)
  & = a_0 + \sum_{k=1}^{\infty} a_k \cos(kx) + b_k \sin(kx)\\
  & = \frac{1}{2\pi} \int_{0}^{2\pi} f(t) \, dt +
  \sum_{k=1}^{\infty} \frac{1}{\pi}\int_{0}^{2\pi} 
  f(t) \cos(t) \, dt
  \cdot \cos(kx) \\
  &\;\;\;\;\; + \sum_{k=1}^{\infty} \frac{1}{\pi} \int_{0}^{2\pi} 
  f(t) \sin(kt)\, dt \cdot \sin(kx)\\
  &= \frac{1}{2\pi} \int_{0}^{2\pi} f(t) \, tx
  + \sum_{k=1}^{\infty} \frac{1}{\pi} \int_{0}^{2\pi} 
  f(t) \cos(k(t-x))\, dt,
\end{align*}
wobei wir bei der letzten Gleichung das Additionstheorem
\[
  \cos(y - z) = \cos(y) \cos(z) + \sin(y) \sin(z)
\]
verwendet haben.
Wir definieren
\begin{align*}
 s_n(x) 
 & = \frac{1}{2\pi} \int_{0}^{2\pi} f(t) \, dt
  + \sum_{k=1}^{n} \frac{1}{\pi}
  \int_{0}^{2\pi} f(t) \cos(k(t-x)) \, dt \\
 &= \frac{1}{\pi} \int_{0}^{2\pi} f(t) D_n(t - x) \, dt,
\end{align*}
wobei
\[
  D_n(t - x) = \frac{1}{2} + \sum_{k=1}^{n} \cos(k(t-x)).
\]
Die Folge ${(D_{n})}_{n \in \mathbb{N}}$ heisst
\emph{Dirichletkern}.
Wir fragen uns nun,
für welche $x \in \mathbb{R}$ die Gleichung
\(
  \lim_{n \to \infty} s_n(x) = f(x)
\)
gilt.
Wir hoffen, dass
\(
  \lim_{n \to \infty} 1/\pi \cdot D_n(t-x) = \delta(t-x),
\)
wobei $\delta$ die ``Deltafunktion'' ist (welche nicht wirklich
eine Funktion ist). Sie erfüllt nämlich $\delta(x) = 0$
für alle $x \neq 0$ und
\[
  \int_{-\infty}^{\infty} \delta(t) \, dt = 1.
\]
Hier gibt es aber einige Probleme.
\begin{enumerate}[(1)]
  \item Der Dirichletkern $D_n$ ist nicht positiv.
  \item Für festes $x \in \mathbb{R}$ und $t \neq x$ 
    ist die Folge ${(D_n(t-x))}_{n \in \mathbb{N}}$
     keine Nullfolge.
     Der Grund dafür ist, dass
     \(
       D_n(t- x) - D_{n-1}(t-x)
       = \cos(n (t-x))
     \) gilt.
\end{enumerate}

Unser Ausweg ist es, nochmals zu mitteln.
Definiere
\[
  \sigma_n(x) = \frac{1}{n} \sum_{k=0}^{n-1} s_k(x),
\]
wobei
\(
  s_0(x) = a_0
\)
konstant ist.
Weiter setze
\[
  F_n(t - x) = \frac{1}{n} \sum_{k=0}^{n-1} D_k(t-x).
\]
Dann heisst ${(F_{n})}_{n \in \mathbb{N}}$ der \emph{Fejérkern}.
Es gilt dann
\[
  \sigma_n(x) = \frac{1}{\pi} \int_{0}^{2\pi} f(t)
  F_n (t-x)\, dt.
\]
Wir können nun Theorem~\ref{thm:fejer} folgendermassen
umformulieren.

\begin{theorem*}[Fejér]
  Sei $f \colon \mathbb{R} \to \mathbb{R}$ 
  stetig und $2\pi$-periodisch. Dann konvergiert
  die Folge ${(\sigma_{n})}_{n \in \mathbb{N}}$ 
  auf $\mathbb{R}$ gleichmässig gegen
  die Funktion $f(x)$.
\end{theorem*}















\end{document}

\documentclass[../main.tex]{subfiles}

\begin{document}
\chapter{Folgen und Reihen}
\begin{definition}
	Eine \textit{Folge} in $\mathbb{R}$ ist eine Abbildung
  \begin{align*}
    a \colon \mathbb{N} & \to \mathbb{R} \\
    n & \mapsto a_n.
  \end{align*}
  Wir notieren das häufig als 
  $(a_0, a_1, a_2, \dots) = {(a_n)}_{n \in \mathbb{N}}$.
\end{definition}

Die für uns wichtigste Eigenschaft von Folgen ist deren Konvergenz.

\begin{definition}
	Eine Folge ${(a_n)}_{n \in \mathbb{N}}$ in $\mathbb{R}$
	heisst \textit{konvergent} mit Grenzwert
	$L \in \mathbb{R}$, falls für alle
	$\varepsilon \in \mathbb{R}$ mit $\varepsilon > 0$
	eine Zahl $N \in \mathbb{N}$ existiert,
	so dass für alle $n\in \mathbb{N}$ 
	mit $n \geq N$ gilt, dass
	$|a_n - L| \leq \varepsilon$. In diesem Fall schreiben wir
	\[
	  \lim_{n \to \infty} a_n = L.
	\]
\end{definition}

In anderen Worten bedeutet $\lim_{n \to \infty} a_n = L$, dass
für vorgegebenes $\varepsilon > 0$
ab einem gewissen Index die Folge für immer im Intervall
$[L- \varepsilon, L + \varepsilon]$ liegt.

\begin{examples}
\begin{enumerate}[(1)]
	\leavevmode
  \item Sei $a_n = 1/n$ für $n \geq 1$. Wir behaupten,
	  dass diese Folge konvergent mit Grenzwert
	  $0 \in \mathbb{R}$ ist. Dazu sei $\varepsilon > 0$
	  vorgegeben.
	  Nach dem Archimedischen Prinzip
	  existiert $N \in \mathbb{N}$ mit
	  $N > 1/\varepsilon$.
	  Dann gilt für alle $n \in \mathbb{N}$ mit $n \geq N$,
	  dass $n > 1/\varepsilon$, also insbesondere
	  nach dem zweiten Ordnungsaxiom, dass
	  $1/n < \varepsilon$.
	  Folglich ist
	  \[
		  |a_n - 0| = \frac{1}{n} < \varepsilon,
	  \]
	  also
	  \[
		  \lim_{n \to \infty}\frac{1}{n} = 0.
	  \]
	  Varianten davon sind
	  \begin{itemize}
	    \item $\lim_{n \to \infty} \frac{1}{n^2} = 0$,
	    \item $\lim_{n \to \infty} \frac{1}{\sqrt n} = 0$,
	    \item $\lim_{n \to \infty} \frac{1}{n^a} = 0$ für
		    eine ``vernünftige'' Potenz $a$.
	  \end{itemize}
  \item Sei $a_n = \sqrt[n]{n}$. Wir behaupten, dass
	  \[
		  \lim_{n \to \infty} \sqrt[n]{n} = 1.
	  \]
	  Es gilt, dass 
	  \[
		  \sqrt[n]{n} \leq 1 + \sqrt{\frac{2}{n-1}}.
	  \]
	  Tatsächlich gilt
	  \[
		  n  \leq \left( 1 + \sqrt{\frac{2}{n-2}} \right)^n,
	  \]
	  denn Anwenden der binomischen Formel
	  \[
		  (a + b)^n = \binom{n}{0}a^n + \binom{n}{1}a^{n-1}b
		  + \dots + \binom{n}{n-1}ab^{n-1} + \binom{n}{n}b^n
	  \]
	  liefert
	  \[
		  \left( 1 + \sqrt{\frac{2}{n-1}} \right)^n
		  = 1 + n \cdot \sqrt{\frac{2}{n-1}}
		  + \binom{n}{2} \left( \sqrt{\frac{2}{n-1}} \right)^2
		  + R
	  \]
	  mit $R \geq 0$.
	  Somit gilt für alle $n \geq 1$ die Ungleichung
	  \[
		  1 \leq \sqrt[n]{n} \leq 1 + \sqrt{\frac{2}{n-1}}.
	  \]
	  Aus
	  \[
		  \lim_{n \to \infty} 1 + \sqrt{\frac{2}{n-1}} = 1
	  \]
	  folgt nun auch
	  \[
		  \lim_{n \to \infty} \sqrt[n]{n} = 1.
	  \]
  \item Sei $q \in \mathbb{R}$ mit $q \geq 0$. Dann gilt
	  \[
	    \lim_{n \to \infty} q^n = 
	    \begin{cases}
		    0, & \text{falls } q < 1, \\
		    1, & \text{falls } q = 1, \\
		    +\infty, & \text{falls } q > 1.
	    \end{cases}
	  \]
	Die Notation
	\[
	  \lim_{n \to \infty} a_n = + \infty
	\]
	heisst, dass für alle $S > 0$ ein Index $N \in \mathbb{N}$
	existiert, so dass für alle $n \geq N$ gilt, dass
	$a_n > S$ ist.
	Der zweite der Fälle ist klar. Für den dritten Fall, 
	betrachte
	\[
		q^n = \left( 1 + (q-1) \right)^n \geq 1 + n(q-1),
	\]
	wobei $q-1 > 0$. 
	Sei $S > 0$ vorgegeben. Wähle $N \in \mathbb{N}$ mit
	\[
	  N \geq \frac{S}{q-1}.
	\]
	Für alle $n \geq N$ gilt dann $q^n \geq 1 + S > S$.
	Für den ersten Fall ersetze $q$ durch $1/q$.
\end{enumerate}
\end{examples}




\end{document}

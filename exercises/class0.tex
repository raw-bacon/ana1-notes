\documentclass[12pt,a4paper]{article}

\usepackage{tikz}
\usepackage{amsmath}
\usepackage{amsfonts}
\usepackage{amsthm}
\usepackage{amsmath}
\usepackage[german]{babel}
\usepackage{enumerate}

\newtheorem*{claim}{Behauptung}
\newtheorem*{lemma}{Lemma}

\theoremstyle{definition}
\newtheorem*{definition}{Definition}
\newtheorem*{examples}{Beispiele}

\title{Übungsstunde 0}
\author{Analysis 1}
\date{18. September 2020}
\begin{document}
\maketitle
\section{Induktion}
\subsection*{Analogie aus dem Alltag}
Induktionsbeweise können für Aussagen
über die natürlichen Zahlen
gebraucht werden.
Sie funktionieren wie Domino
in folgendem Sinn.
Den Stein, den man zu Beginn anstösst,
ist die \textit{Verankerung}.
Die Tatsache, dass jeder Stein
den folgenden Stein zum Umfallen bringt,
ist der \textit{Induktionsschritt}.
Nehmen wir an, dass
\begin{itemize}
  \item der erste Stein angestossen wird,
  \item das Umfallen eines Steins das Umfallen des nächsten verursacht,
\end{itemize}
können wir schliessen, dass alle Steine umfallen.

\subsection*{Mathematische Interpretation}
Sei $A_{n}$ eine Aussage über eine natürliche Zahl $n \in \mathbb N$.
In der Mathematik heisst das, dass $A_n$
entweder wahr oder falsch ist.
Interpretieren wir das Umfallen des Dominosteins $n$ oben
als ``$A_{n}$ ist wahr'', liefert das folgende Erkenntnis.
Falls wir für ein $n_{0} \in \mathbb N$ zeigen können,
dass
\begin{itemize}
  \item
    $A_{n_{0}}$ wahr ist, und
  \item
    wenn immer $A_{n}$ wahr ist, dann auch $A_{n+1}$,
\end{itemize}
so gilt $A_{n}$ für alle $n \geq n_{0}$.
Dieses Prinzip nennt man ``Induktion''.

\subsection*{Beispiele}
\begin{claim}
  Sei $n \in \mathbb N$. Dann gilt
  \[1 + 2 + \cdots + n = \frac{n(n+1)}{2}.\]
\end{claim}

\begin{proof}
  Wir machen die Verankerung bei $n = 2$, da das
  die erste nicht-pathologische Aussage ist.
  Interpretieren wir die Summe links richtig,
  so stimmt die Behauptung aber auch für $n= 0$ und $n = 1$.
  Berechne
  \[1 + 2 = \frac{2(2+1)}{2}.\]
  Also stimmt die Behauptung für $n = 2$. Nehmen wir nun an, dass
  \[1 + 2 + \cdots + n = \frac{n(n+1)}{2}\]
  für ein bestimmtes $n \in \mathbb N$, berechne
  \begin{align*}
    1 + 2 + \cdots + n + (n+1) &= \frac{n(n+1)}{2} + (n+1) \\
                               &= \frac{n^2 + n + 2n + 2}{2} \\
                               &= \frac{(n+1)(n+2)}{2}.
  \end{align*}
  Das ist aber genau was wir behauptet haben,
  nur dass wir $(n+1)$ für $n$ eingesetzt haben.
\end{proof}

\begin{claim}
  Ein Brett der Grösse $2^{n} \times 2^{n}$
  kann mit Steinen der Form
  \[
    \begin{tikzpicture}[scale=0.4]
      \draw[gray] (0, 1) -- (1, 1) -- (1, 2);
      \draw (0, 0) -- (0, 2) -- (2, 2)
      -- (2, 1) -- (1, 1)
      -- (1, 0) -- cycle;
    \end{tikzpicture}
  \]
  bedeckt werden, so dass genau ein Feld frei bleibt.
\end{claim}

\begin{figure}[htb]
  \centering
  \begin{minipage}{0.33\linewidth}
    \centering
    \begin{tikzpicture}[scale=1.2]
      \draw (0, 0) rectangle (2, 2);
      \foreach \x in {1, ..., 1} {
        \draw[ultra thin, gray] (0, \x) -- (2, \x);
        \draw[ultra thin, gray] (\x, 0) -- (\x, 2);
      }
      \fill (0, 0) rectangle (1, 1);
    \end{tikzpicture}
  \end{minipage}%
  \begin{minipage}{0.33\linewidth}
    \centering
    \begin{tikzpicture}[scale=0.6]
      \draw (0, 0) rectangle (4, 4);
      \foreach \x in {1, ..., 3} {
        \draw[ultra thin, gray] (0, \x) -- (4, \x);
        \draw[ultra thin, gray] (\x, 0) -- (\x, 4);
      }
      \draw (2, 0) -- (2, 4);
      \draw (0, 2) -- (4, 2);
      \fill (0, 0) rectangle (1, 1);
      \fill (2, 2) rectangle (3, 3);
      \fill (2, 1) rectangle (3, 2);
      \fill (1, 2) rectangle (2, 3);
    \end{tikzpicture}
  \end{minipage}%
  \begin{minipage}{0.33\linewidth}
    \centering
    \begin{tikzpicture}[scale=0.3]
      \draw (0, 0) rectangle (8, 8);
      \foreach \x in {1, ..., 7} {
        \draw[ultra thin, gray] (0, \x) -- (8, \x);
        \draw[ultra thin, gray] (\x, 0) -- (\x, 8);
      }
      \draw (4, 0) -- (4, 8);
      \draw (0, 4) -- (8, 4);
      \fill (4, 4) rectangle (5,5);
      \fill (3, 4) rectangle (4, 5);
      \fill (4, 3) rectangle (5, 4);
      \fill (0, 0) rectangle (1, 1);
    \end{tikzpicture}
  \end{minipage}%
  \begin{minipage}{0.5\linewidth}
    \centering
  \end{minipage}
  \caption{Verankerung und zwei Induktionsschritte}%
  \label{fig:induktion}
\end{figure}

\begin{proof}[Beweisskizze]
  Die Verankerung kann man hier für $n = 0$ machen, aber
  auch hier ist das etwas pathologisch. Wir machen die
  Verankerung bei $n = 1$ wie in Abbildung~\ref{fig:induktion}.
  Wichtig ist hier zu bemerken, dass wir das so
  bewerkstelligen können, dass das freie Feld
  in einer Ecke liegt.

  Für den Induktionsschritt, das heisst das Bedecken eines
  Bretts der Grösse $2^{n+1} \times 2^{n+1}$,
  nehmen wir zuerst vier mal bedeckte Bretter der
  Grösse $2^{n} \times 2^{n}$.
  Nach richtigem rotieren dieser
  kleineren Bretter
  erhalten wir etwas
  wie in Abbildung~\ref{fig:induktion}.
  Wir sehen, dass ein isoliertes Feld in der Ecke frei bleibt,
  und drei in der Mitte in der Form der Steine
  die wir zur Verfügung haben.
  Bedecken wir diese drei Felder
  erhalten wir die gesuchte Konstellation.
\end{proof}

\section{Binomialkoeffizienten}
\begin{definition}
  Seien $k \leq n$ natürliche Zahlen.
  Der \textit{Binomialkoeffizient} $\binom{n}{k}$ bezeichnet die Anzahl
  Art und Weisen, $k$ Elemente aus $n$ Elementen auszuwählen.
\end{definition}

\begin{examples}
  Ohne Formel können wir aus der Definition herauslesen, dass
  \begin{itemize}
    \item $\binom{n}{0} = 1$
    \item $\binom{n}{1} = n$
    \item $\binom{n}{n} = 1$
    \item $\binom{n}{n-1} = n$
    \item $\binom{n}{k} = \binom{n}{n-k}$
  \end{itemize}
\end{examples}

\begin{lemma}[Pascal]
  Seien $k \leq n$ natürliche Zahlen. Dann gilt
  \[
    \binom{n+1}{k} = \binom{n}{k} + \binom{n}{k-1}.
  \]
\end{lemma}

\begin{proof}
  Fixiere ein Element $x$. Jede Art und Weise,
  $k$ Elemente aus $(n+1)$ Elementen auszuwählen,
  fällt in eine der folgenden Fälle.
  \begin{enumerate}[(a)]
    \item Unsere Auswahl lässt $x$ aus.
    \item Unsere Auswahl enthält $x$.
  \end{enumerate}
  Fall (a) liefert $\binom{n}{k}$ Möglichkeiten,
  und Fall (b) liefert $\binom{n}{k-1}$ Möglichkeiten.
\end{proof}

Dieses Lemma zeigt, dass wir Binomialkoeffizienten
in Form des \textit{Pascalschen Dreieck}
anordnen können. Siehe Abbildung~\ref{fig:pascal}.

\begin{figure}[htb]
  \centering
  \begin{tikzpicture}[yscale=-1]
    \node at (0, 0) {$\binom{0}{0}$};

    \node at (-1, 1) {$\binom{1}{0}$};
    \node at (1, 1) {$\binom{1}{1}$};
    \draw (-0.5, 1) -- (0.5, 1);
    \draw (0, 1) -- (0, 1.5);

    \node at (-2, 2) {$\binom{2}{0}$};
    \node at (0, 2) {$\binom{2}{1}$};
    \node at (2, 2) {$\binom{2}{2}$};
    \draw (-1.5, 2) -- (-0.5, 2);
    \draw (0.5, 2) -- (1.5, 2);
    \draw (1, 2) -- (1, 2.5);
    \draw (-1, 2) -- (-1, 2.5);
   
    \node at (-3, 3) {$\binom{3}{0}$};
    \node at (-1, 3) {$\binom{3}{1}$};
    \node at (1, 3) {$\binom{3}{2}$};
    \node at (3, 3) {$\binom{3}{3}$};
    \draw (-2.5, 3) -- (-1.5, 3);
    \draw (-0.5, 3) -- (0.5, 3);
    \draw (1.5, 3) -- (2.5, 3);
    \draw (-2, 3) -- (-2, 3.5);
    \draw (0, 3) -- (0, 3.5);
    \draw (2, 3) -- (2, 3.5);

    \node at (-4, 4) {$\binom{4}{0}$};
    \node at (-2, 4) {$\binom{4}{1}$};
    \node at (0, 4) {$\binom{4}{2}$};
    \node at (2, 4) {$\binom{4}{3}$};
    \node at (4, 4) {$\binom{4}{4}$};
  \end{tikzpicture}
  \caption{Pascalsches Dreieck}%
  \label{fig:pascal}
\end{figure}

\begin{claim}
  Seien $k \leq n$ natürliche Zahlen. Dann gilt
  \[\binom{n}{k}
  = \frac{n!}{k!(n-k)!}.\]
\end{claim}

\begin{proof}
  Wir zeigen diese Behauptung unter Verwendung des obigen Lemmas
  mit Induktion.
  Die Verankerung bei $n = 0$ hatten wir bereits als Beispiel oben.
  Für den Induktionsschritt, berechne:
  \begin{align*}
    \binom{n+1}{k} &= \binom{n}{k} + \binom{n}{k-1} \\
                   &= \frac{n!}{k!(n-k)!} + \frac{n!}{(k-1)!(n-k+1)!} \\
                   &= \frac{n!}{k(k-1)!(n-k)!} + \frac{n!}{(k-1)!(n-k)!(n-k+1)} \\
                   &= \frac{(n-k+1)n! + kn!}{k(k-1)!(n-k)!(n-k+1)} \\
                   &= \frac{(n+1)!}{(k+1)!(n-k+1)!}.
  \end{align*}
  Das ist genau was wir zeigen wollten.
\end{proof}
\end{document}
